\documentclass[12pt]{article}
\setlength{\oddsidemargin}{0.0in} \setlength{\evensidemargin}{0.0in}
\setlength{\textwidth}{6.5in} \setlength{\topmargin}{-0.25in}
\setlength{\textheight}{8.75in}
\usepackage{amsmath, amssymb, amsfonts, amsthm, amscd, xspace, pifont, natbib, fullpage, enumitem, bm, bbm}
\usepackage{fullpage}
\usepackage{graphicx, float}
\usepackage{epsfig, amsfonts, verbatim, multirow, hyperref}
\usepackage{epstopdf}
\usepackage{listings, boxedminipage}

\lstset{
	basicstyle=\ttfamily,
	mathescape
}
%\usepackage{setspace}
\newcommand{\mycite}[1]{{\citeNP{#1}}}

\parindent=0pt

%%%%%%%%%%%%%%%%%%%%%%%%%%%%%%%%%%%%%%%%%%%%%%%%%%%%%%%%%%%%%%%%%%%%%%%%%%%%%%%%%%%%%%%%%%%%%%%%%%%
% Different font in captions
\newcommand{\captionfonts}{\small}
\makeatletter  % Allow the use of @ in command names
\long\def\@makecaption#1#2{%
	\vskip\abovecaptionskip
	\sbox\@tempboxa{{\captionfonts #1: #2}}%
	\ifdim \wd\@tempboxa >\hsize
	{\captionfonts #1: #2\par}
	\else
	\hbox to\hsize{\hfil\box\@tempboxa\hfil}%
	\fi
	\vskip\belowcaptionskip}
\makeatother   % Cancel the effect of \makeatletter
%%%%%%%%%%%%%%%%%%%%%%%%%%%%%%%%%%%%%%%%%%%%%%%%%%%%%%%%%%%%%%%%%%%%%%%%%%%%%%%%%%%%%%%%%%%%%%%%%%%

%% Matrix, Vector
\newcommand{\V}[1]{\ensuremath{\boldsymbol{#1}}\xspace}
\newcommand{\M}[1]{\ensuremath{\boldsymbol{#1}}\xspace}
%% Math Functions
\newcommand{\F}[1]{\ensuremath{\mathrm{#1}}\xspace}
\newcommand{\sgn}{\F{sgn}}
\newcommand{\tr}{\F{trace}}
\newcommand{\diag}{\F{diag}}
\newcommand{\dett}{\F{det}}
%% Transpose
\newcommand{\T}[1]{\ensuremath{{#1}^{\mbox{\sf\tiny T}}}}

%%
\def  \R  {\boldsymbol R}
\def\bX{\boldsymbol X}
\def\bY{\boldsymbol Y}
\def\bbeta{\boldsymbol \beta}
\def\blambda{\boldsymbol \lambda}
\def\bepsilon{\boldsymbol \epsilon}
\def\bone{\boldsymbol{1}}
\def\bzero{\boldsymbol 0}
\def\E{\mbox{E}}
\def\var{\mbox{var}}
\def\gauss{\mbox{N}}
\def\lap{\mbox{L}}
\def\G{\mbox{G}}
\def\go{  $\R$  ightarrow}
\def\invG{\mbox{G}^{-1}}
\def\argmin{\arg\min}

\newtheorem{theorem}{Theorem}
\newtheorem{proposition}{Proposition}
\newtheorem{lemma}{Lemma}
\newtheorem{corollary}{Corollary}
\newtheorem{algorithm}{Algorithm}
\newtheorem{definition}{Definition}
\newtheorem{remark}{Remark}

\usepackage{color}
\usepackage[dvipsnames]{xcolor}
\definecolor{codegreen}{rgb}{0,0.6,0}
\definecolor{codegray}{rgb}{0.5,0.5,0.5}
\definecolor{codepurple}{rgb}{0.58,0,0.82}
\definecolor{backcolour}{rgb}{0.95,0.95,0.92}
\lstdefinestyle{displaycode}{
	backgroundcolor=\color{backcolour},
	commentstyle=\color{codegreen},
	keywordstyle=\color{magenta},
	numberstyle=\tiny\color{codegray},
	stringstyle=\color{codepurple},
	basicstyle=\footnotesize,
	breakatwhitespace=false,
	breaklines=true,
	captionpos=b,
	keepspaces=true,
	columns=flexible,
	numbers=none,
	numbersep=5pt,
	showspaces=false,
	showstringspaces=false,
	showtabs=false,
	tabsize=2%,
	%xleftmargin=2em,
	%xrightmargin=2em,
}
\lstdefinestyle{displaycode2}{
	backgroundcolor=\color{backcolour},
	commentstyle=\color{red},
	keywordstyle=\color{magenta},
	numberstyle=\tiny\color{codegray},
	stringstyle=\color{codepurple},
	basicstyle=\footnotesize,
	breakatwhitespace=false,
	breaklines=true,
	captionpos=b,
	keepspaces=true,
	columns=flexible,
	numbers=none,
	numbersep=5pt,
	showspaces=false,
	showstringspaces=false,
	showtabs=false,
	tabsize=2%,
	%xleftmargin=2em,
	%xrightmargin=2em,
}

\lstset{style=displaycode}
%\newcommand{\displaycodefile}[1]{\lstinputlisting[language=R]{./codeblocks/#1.txt}}

\newcommand{\pr}{\mathbb{P}}
\newcommand{\ep}{\mathbb{E}}
\renewcommand{\var}{\textrm{Var}}

\renewcommand{\thefootnote}{\fnsymbol{footnote}}
\newcommand{\x}{\bm{x}}
\newcommand{\X}{\bm{X}}
\newcommand{\td}{\textrm{d}}

\newcommand{\tblue}[1]{\textcolor{blue}{#1}}
\newcommand{\tred}[1]{\textcolor{red}{#1}}
\newcommand{\tplum}[1]{\textcolor{Plum}{#1}}
\newcommand{\tcyan}[1]{\textcolor{Cyan}{#1}}
\newcommand{\taqua}[1]{\textcolor{Aquamarine}{#1}}

\newcommand{\CAL}[1]{{\cal #1}}

\begin{document}
\title{\Large \bf STATS 406 Fall 2017: Lab SQL}
\date{}

\maketitle

\section{Introduction to relational databases}
\begin{itemize}
	\item The advantage of using relational databases:
	\begin{itemize}[label=*]
		\item Data can be maintained and queried by different threads independently.
		\item More efficient queries for objectives that only concerns one subtable.
	\end{itemize}
	\item Relational database files and their management systems:
	\begin{itemize}[label=*]
		\item A .db file is to relational database management systems (SQLite, MySQL, Oracle, etc) as a .pdf file is to PDF readers (Adobe, Foxit, Okular, etc).
		\item There is one file format (.db) and many tools (management systems) you can use to manage the file.
		\item Different management systems have very similar syntaxes on basic operations and return queried data in very similar forms. They differ in efficiency and other aspects, but not much grammatically.
		\item In fact, the commands and even most code lines you learned in this class will run without modification under most management systems.
	\end{itemize}
\end{itemize}

\section{Using SQLite in R}
\begin{itemize}
	\item We can work with databases within R by installing related packages.
	\item Install package RSQLite and connect to a database.
\begin{lstlisting}[style=displaycode, language=R]
## Load the RSQLite package (install it first if haven't yet)
install.packages("RSQLite", dep=TRUE);
library("RSQLite");
## Make sure the file is under the working directory and then connect to it
driver = dbDriver("SQLite");
conn = dbConnect(driver, "baseball.db");
## Check the tables in a database
dbListTables(conn);
## Check the variables in a table
dbListFields(conn, "Teams");
\end{lstlisting}

\end{itemize}


\section{SQL commands}

\subsection{A few notes}
\begin{itemize}
	\item SQL keywords are not case sensitive, but when calling SQL in R, capitalizing key words helps to improve readability.
	\item SQL is designed mainly for extracting data from databases, NOT for analyzing them. SQL provides basic summarizing commands and tools, but do not expect too much. We can use R for further analysis.
\end{itemize}

\subsection{Basic SQL commands:}
\begin{itemize}
	\item The very basic form of SQL queries is:
\begin{lstlisting}[style=displaycode, language=SQL]
/* Pseudo-code */
/* Required clauses */
SELECT ColumnNames
FROM TableName
/* Optional clauses */
WHERE Conditions
GROUP BY VariableNames HAVING Conditions
ORDER BY VariableNames
\end{lstlisting}
	In {\bf SELECT}:
	\begin{itemize}[label=*]
		\item {\bf AS} renames the extracted variables for convenience. It is especially useful when a. they are summarized; or b. (will see later) when they have to come with prefixes like table names.
		\item {\bf AS} can often be omitted. Thus the variable name should NOT contain space (Why?).
		\item Aggregate functions. For a list, check, for example, \url{http://www.techonthenet.com/sql_server/functions/index_alpha.php}.
\begin{lstlisting}[style=displaycode, language=SQL]
/* Pseudo-code: SQLite */
SELECT Var1, MIN(Var2) AS MinVar2, AVG(Var3) AS AvgVar3
FROM TableName
\end{lstlisting}
	\end{itemize}
	In {\bf FROM}:
	\begin{itemize}[label=*]
		\item In this course, unless we combine tables, otherwise we only select from one table.
	\end{itemize}
	In {\bf WHERE}:
	\begin{itemize}[label=*]
		\item If there are multiple conditions, they should be connected by logical connectives ({\bf AND}, {\bf OR} and parenthesis when needed). Conditions can also be decorated by other logical operators ({\bf NOT}, etc). For a list of logical operators in SQL, see \url{http://www.w3resource.com/sql/boolean-operator/sql-boolean-operators.php}.
\begin{lstlisting}[style=displaycode, language=SQL]
/* Pseudo-code: SQLite */
/* Example from: http://beginner-sql-tutorial.com/sql-logical-operators.htm */
SELECT first_name, last_name, age, games
FROM student_details
WHERE age >= 10 AND age <= 15 OR NOT games = 'Football'
\end{lstlisting}
	\end{itemize}
{\bf Quiz 1:} Extract data from table \emph{Master}. Generate a table with all variables in Master for players born in 1980s and sort by their names.
\begin{lstlisting}[style=displaycode, language=SQL]
# Solution: see Lab4.R
\end{lstlisting}
	\item For {\bf GROUP BY}:
	\begin{itemize}[label=*]
		\item {\bf GROUP BY} is used in combination with aggregate functions in {\bf SELECT}.
		\item (From Wikipedia) {\bf HAVING} modifies {\bf GROUP BY}. It is indispensable because {\bf WHERE} does not allow aggregate functions.
\begin{lstlisting}[style=displaycode, language=SQL]
/* Pseudo-code: SQLite */
/* Example from: https://en.wikipedia.org/wiki/Having_(SQL) */
SELECT DeptID, SUM(SaleAmount)
FROM Sales
WHERE SaleDate = '01-Jan-2000'
GROUP BY DeptID
HAVING SUM(SaleAmount) > 1000
\end{lstlisting}
	\end{itemize}
\end{itemize}

{\bf Quiz 2:} Query data from table \emph{Teams}. Generate a table showing the number of teams that won more than half of the games each year since 1950. Sort by year.

\begin{lstlisting}[style=displaycode, language=SQL]
# Solution: see Lab4.R
\end{lstlisting}

\subsection{Inner joining tables}
\begin{itemize}
	\item Basic form (not quite ``basic'', look closely):
\begin{lstlisting}[style=displaycode, language=SQL]
/* Pseudo-code */
SELECT T1.ColumnNames, T2.ColumnNames
FROM T1 INNER JOIN T2 ON JoiningConditions
/* Other clauses */
WHERE Conditions
/* etc */
\end{lstlisting}
	\item What are the different roles of Table ``T1'' and Table ``T2''?
	\item {\bf ON}: Specify which rows to combine from the two tables
	\item In {\bf SELECT} here: specify table names as prefixes for variable names when needed.
	\item \underline{\bf Example:}
\begin{lstlisting}[style=displaycode, language=SQL]
SELECT Schools.schoolID, schoolName, schoolCity, schoolState, playerID
FROM Schools INNER JOIN SchoolsPlayers ON Schools.schoolID = SchoolsPlayers.schoolID
WHERE schoolState = 'MI'
\end{lstlisting}
\begin{itemize}[label=*]
\item If we omit ``Schools'' in the first ``Schools.schoolID'', there will be an error.
\item In general, DO NOT omit table names if causing ambiguity.
\end{itemize}
\end{itemize}
{\bf Quiz 3:} Join table \emph{Schools} with table \emph{SchoolsPlayers}. Generate a table showing the number of players from each school in Michigan. Sort by the number of players in decreasing order.

\begin{lstlisting}[style=displaycode, language=SQL]
# Solution: see Lab4.R
\end{lstlisting}
\end{document}